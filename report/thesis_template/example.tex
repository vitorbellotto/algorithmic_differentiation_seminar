%******************************************************************************%
% EXEMPLARY USAGE OF THE DEFINED COMMANDS
%******************************************************************************%

% !TEX program = pdflatex or xelatex
%% !BIB program = bibtex or biber or something else

%******************************************************************************%
% CLASS SETTINGS
%******************************************************************************%
\documentclass[
  a4paper, 
  fontsize=12pt, 
  titlepage, 
  listof=totocnumbered, 
  index=totoc,        
  bibliography=totocnumbered, 
  numbers=noenddot
]{scrartcl}
%******************************************************************************%

%******************************************************************************%
% STYLE SPECIFICATION (choose english or german)
%******************************************************************************%
\usepackage[english]{template/thesisstyle} 
%******************************************************************************%

%******************************************************************************%
% GUIDE
%******************************************************************************%
\begin{document}

% Math commands
\section{Math Commands}

  \begin{tabular}{l  l  c}
    \toprule
      Description & LateX Command & Result \\[3pt]
    \midrule
      Absolute value & $\backslash \text{abs}\lbrace \text{x} \rbrace$ & $\abs{x}$ \\[3pt]
      Norm & $\backslash \text{norm}\lbrace \text{x} \rbrace\_2$ & $\norm{x}_2$ \\[3pt]
      Ceil & $\backslash \text{ceil}\lbrace \text{x} \rbrace$ & $\ceil{x}$ \\[3pt]
      Floor & $\backslash \text{floor}\lbrace \text{x} \rbrace$ & $\floor{x}$ \\[3pt]
      Scalar Product & $\backslash \text{scalarp}\lbrace \text{x} \rbrace$ & $\scalarp{x}$ \\[3pt]
      Average & $\backslash \text{avg}( \text{x} )$ & $\avg(x)$ \\[3pt]
      Sign & $\backslash \text{sign}( \text{x} )$ & $\sign(x)$ \\[3pt]
      Variance & $\backslash \text{var}( \text{x} )$ & $\var(x)$ \\[3pt]
      Diagonal & $\backslash \text{diag}( \text{x} )$ & $\diag(x)$ \\[3pt]
      Divergence & $\backslash \text{diver}( \text{x} )$ & $\diver(x)$ \\[3pt]
      Gradient & $\backslash \text{grad}( \text{x} )$ & $\grad(x)$ \\[3pt]
      Landau symbol & $\backslash \text{landau}( \text{n} )$ & $\landau(n)$ \\[3pt]
      \midrule
      For all & $\backslash \text{fa}$ & $\fa$ \\[3pt]
      Follows & $\backslash \text{fol}$ & $\fol$ \\[3pt]
      Is & $\backslash \text{is}$ & $\is$ \\[3pt]
      Must be equal & $\backslash \text{mbeq}$ & $\mbeq$ \\[3pt]
      Must be lower & $\backslash \text{mbl}$ & $\mbl$ \\[3pt]
      Must be greater & $\backslash \text{mbg}$ & $\mbg$ \\[3pt]
      Partial derivative & $\backslash \text{pdv}\lbrace \text{f(x)} \rbrace\lbrace \text{x} \rbrace$ & $\pdv{f(x)}{x}$ \\[3pt]
      Infinitesimal symbol & $\backslash \text{dd}$ & $\dd$ \\[3pt]
      Total derivative & $\backslash \text{tdv}\lbrace \text{f(x)} \rbrace\lbrace \text{x} \rbrace$ & $\tdv{f(x)}{x}$ \\[3pt]
      Times with less space & $3\backslash \text{timesSmall }3$ & $3\timesSmall 3$ \\[3pt]
      \midrule
      Maximum & $\backslash \text{mmax}\lbrace \text{p} \rbrace \sim\text{f(x,p)}$ & $\mmax{p}~f(x,p)$ \\[3pt]
      Minimum & $\backslash \text{mmin}\lbrace \text{p} \rbrace \sim\text{f(x,p)}$ & $\mmin{p}~f(x,p)$ \\[3pt]
      Supremum & $\backslash \text{msup}\lbrace \text{p} \rbrace \sim\text{f(x,p)}$ & $\msup{p}~f(x,p)$ \\[3pt]
      Infimum & $\backslash \text{minf}\lbrace \text{p} \rbrace \sim\text{f(x,p)}$ & $\minf{p}~f(x,p)$ \\[3pt]
      Limit & $\backslash \text{mlim}\lbrace \text{p} \rbrace \sim\text{f(x,p)}$ & $\mlim{p}~f(x,p)$ \\[3pt]
      Integral & $\backslash \text{mint}\lbrace\text{f(x)}\rbrace\lbrace\text{x}\rbrace\lbrace\text{a}\rbrace\lbrace\text{b}\rbrace$ & $\mint{f(x)}{x}{a}{b}$ \\[3pt]
      Line Integral & $\backslash \text{moint}\lbrace\text{f(x)}\rbrace\lbrace\text{x}\rbrace\lbrace\text{a}\rbrace\lbrace\text{b}\rbrace$ & $\moint{f(x)}{x}{a}{b}$ \\[3pt]
      \bottomrule
  \end{tabular}
  \newpage
  
\section{Enumerations Commands}
  \paragraph{Roman} 
    $\backslash \text{begin}\lbrace\text{enumRom}\rbrace \backslash \text{item bla}\backslash \text{end}\lbrace\text{enumRom}\rbrace$
    \begin{enumRom} 
      \item bla
      \item bla
      \item bla
    \end{enumRom}

  \paragraph{Arabic} 
  $\backslash \text{begin}\lbrace\text{enumArab}\rbrace \backslash \text{item bla}\backslash \text{end}\lbrace\text{enumArab}\rbrace$
    \begin{enumArab} 
      \item bla
      \item bla
      \item bla
    \end{enumArab}

  \paragraph{Alphabetical} 
  $\backslash \text{begin}\lbrace\text{enumAlph}\rbrace \backslash \text{item bla}\backslash \text{end}\lbrace\text{enumAlph}\rbrace$
    \begin{enumAlph} 
      \item bla
      \item bla
      \item bla
    \end{enumAlph}
  
\section{Coding Commands}
  \paragraph{\cpp} $\backslash \text{cpp}$  
  \paragraph{\dcocpp{}} $\backslash \text{dcocpp}\lbrace\rbrace$  
  \paragraph{\dcofortran{}} $\backslash \text{dcofortran}\lbrace\rbrace$   

\section{Miscellaneous}
  Use command $\backslash \text{begin}\lbrace\text{definition}\rbrace\left[\text{Bla}\right] \text{ My awesome definition!} \backslash \text{end}\lbrace\text{definition}\rbrace$ for your own definitions. They will be listed in the theorem section.

  \begin{definition}[Bla]
    My awesome definition!
  \end{definition}

  \par Use command $\backslash \text{sect}$, $\backslash \text{app}$, $\backslash \text{fig}$, $\backslash \text{tab}$, $\backslash \text{eq}$, $\backslash \text{defi}$ in order to reference a Section, Appendix, Figure, Table, Equation or Definition.



  
\end{document}
