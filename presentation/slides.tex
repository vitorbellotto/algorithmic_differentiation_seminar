 % This file provides an example Beamer presentation using the RWTH theme
% showcasing some of the more common options, similar to the Powerpoint version
% 12.11.2014: Revision 1 (Harold Bruintjes, Tim Lange)

% For RWTH, beamer should be loaded with class option t (top)
\documentclass[t]{beamer}

% Use fontspec to get Arial font
% Requires use of XeLaTeX
\usepackage{fontspec}
\setmainfont{Arial}
\setsansfont{Arial}
% Also force Arial for math for a more consistent look
\usepackage{unicode-math}
%\setmathfont{Arial}
\usepackage{listings}
\lstloadlanguages{[ISO]C++}
\lstset{
        %frame=top,frame=bottom,
        xleftmargin=1.2ex,
        tabsize=2,
        showspaces=false,
        showstringspaces=false,
        captionpos=b,
        language=[ISO]C++,
        basicstyle=\Large\ttfamily, 
        numbers=left, 
        numberstyle=\small, 
        stepnumber=1, 
        numbersep=5pt, 
        breaklines=true,
        escapeinside={/*}{*/}
       }

\lstset{morekeywords={
        then, pragma, omp, parallel, pragma, num_threads, 
        private, firstprivate, lastprivate, shared, default, 
        reduction, copyin, atomic, task, taskyield, master, 
        critical, barrier, taskwait, flush, ordered, threadprivate, 
        ga1s, gt1s, gt1v, ga1v}
       }


\newcommand{\R}{I\!\!R}
\newcommand{\X}{{\bf x}}
\newcommand{\LA}{{\bf \lambda}}
\newcommand{\A}{{\bf a}}
\newcommand{\Y}{{\bf y}}
\newcommand{\B}{{\bf b}}
\newcommand{\bY}{\Y_{(1)}}
\newcommand{\bX}{{\bar\X}}
\newcommand{\dY}{\Y^{(1)}}
\newcommand{\dX}{\X^{(1)}}
\newcommand{\tG}{\tilde{G}}
\newcommand{\tE}{\tilde{E}}
\newcommand{\tV}{\tilde{V}}
\newcommand{\cprec}{\prec^*}
\newcommand{\V}{\mathbb{V}}
\newcommand{\bV}{\overline{\V}}
\newcommand{\hV}{\hat{\V}}
\newcommand{\bhV}{\overline{\hV}}
\newcommand{\dcocpp}{{\ttfamily dco\kern-.08em{\raisebox{-.1ex}{/}\kern-.15em {c\kern-.03em{\raisebox{-.18ex}{+\kern-.028em{+}}}}}}}

% German style date formatting (footer)
\usepackage[ddmmyyyy]{datetime}
\renewcommand{\dateseparator}{.}

\usepackage{MnSymbol,wasysym}

% Format the captions used for figures etc.
\usepackage[compatibility=false]{caption}
\captionsetup{singlelinecheck=off,justification=raggedleft,labelformat=empty,labelsep=none}

% PGFPlots is used for drawing some of the charts
\usepackage{pgfplots}
\pgfplotsset{compat=newest}
\input{plot_commands.tex}

% Load the actual RWTH theme. Suggested is to load the full theme,
% as it requires some specific dimensions
\usetheme{rwth}

\AtBeginSection[]
{
  \begin{frame}<beamer>
    %\frametitle{Outline for section \thesection}
    \tableofcontents[currentsection]
  \end{frame}
}

\begin{document}

\logo{\includegraphics{logo.png}}

% Setup presentation information
\title{Computational Finance using Algorithmic Differentiation}
\date{May 8th 2019}
\author{Vitor Hugo Bellotto Zago}

\frame{\titlepage}

\lstset{language=C++}
\lstset{basicstyle=\small}
\lstset{numbers=none}
\lstset{numberstyle=\scriptsize}
\lstset{numbersep=5pt}

% Frame with items
\begin{frame}
\frametitle{Table of Contents}
  %\dcocpp{}
%\tableofcontents[currentsection]
\tableofcontents[currentsection]
\end{frame}



\section{Introduction}
\begin{frame}Here the goal of the presentation is going to be stated. \end{frame}


\subsection{Manual differentiation}
\begin{frame}
	\frametitle{Manual Differentiation} Image of someone derivating. Time image. 
\end{frame}

\subsection{Symbolic Differentiation}
\begin{frame}
	\frametitle{Symbolic Differentiation} 
	Output differentiation from SymPy. Plot demonstrating the problem called expression swell.
\end{frame}

\subsection{Finite Differentiation}
\begin{frame}
	\frametitle{Finite Differentiation} 
	Central Difference. Plot demonstrating round-off errors. 
\end{frame}

\subsection{Algorithmic Differentiation}
\begin{frame}
	\frametitle{Algorithmic Differentiation}
 	Code chuck with function. Derivative next to it. 
\end{frame}

\section{Categories of AD}
% Forward method
\subsection{Forward Method}
\begin{frame}\frametitle{Forward Method} Definition.\end{frame}
% Reverse method
\subsection{Reverse Method}
\begin{frame}\frametitle{Reverse Method} Definition.\end{frame}

\begin{frame}\frametitle{Advantages and drawbacks}
 Based on the definitions of tangent and adjoint, explain the advantages and disadvantages of AD.
\end{frame}

\section{General Application}
\begin{frame} Machine learning. Optimization problems. Finances. \end{frame}

\section{Finance with AD}
\begin{frame}\frametitle{Libor Market Model} Using finance with AD \end{frame}

\begin{frame}\frametitle{Greeks} \end{frame}

\begin{frame}\frametitle{Monte Carlo pricing?} \end{frame}

\begin{frame}\frametitle{dco++}\end{frame}

\begin{frame}\frametitle{Other methods} \end{frame}

\section{Bibliography}
\begin{frame} Cite books and papers. \end{frame}






\end{document}
